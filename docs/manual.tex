\documentclass[a4paper, 12p, titlepaget]{book}
\usepackage{cmbright}
\usepackage{graphicx}
\usepackage[intlimits]{amsmath}
\usepackage{amssymb}
\usepackage{amsthm}
\usepackage{parskip}
\usepackage{pgf}
\usepackage{tikz}
\usetikzlibrary{shapes}
\usetikzlibrary{positioning}
\usepackage{enumerate}
\usepackage{framed}

\newcommand \name {NESIZER }
\newcommand {\btn}[1] {\framebox{\textbf{\footnotesize #1}}}
\newcommand {\cbtn}[1] {\tikz \node[draw,circle]{\textbf{\footnotesize #1}};}
\newcommand {\lbl}[1] {\emph{\footnotesize #1}}
\newcommand \dsp \texttt


\title{NESIZER}
\author{}
\date{}


\begin{document}

\definecolor{buttonredh}{RGB}{255,0,0}
\definecolor{buttonred}{RGB}{240, 161, 161}
\definecolor{buttongray}{RGB}{71,71,71}

\tikzstyle{mbutton}=[font=\tiny,rectangle, draw, fill=buttonred, text=gray, minimum height=1.2cm, minimum width=1.2cm]
\tikzstyle{mhbutton}=[font=\tiny,rectangle, draw, fill=buttonredh, minimum height=1.2cm,minimum width=1.2cm]

\tikzstyle{hbutton}=[font=\tiny,rectangle, draw, fill=black, text=white, minimum height=1.2cm, minimum width=1.2cm]
\tikzstyle{button}=[font=\tiny,rectangle, draw, fill=buttongray, text=gray, minimum height=1.2cm,minimum width=1.2cm]

\tikzstyle{dummybutton}=[font=\tiny,rectangle, fill=white, text=white, minimum height=1cm,minimum width=1cm]
\tikzstyle{hcircbutton}=[font=\tiny,circle, fill=black, text=white, minimum height=1.2cm,minimum width=1.2cm]
\tikzstyle{circbutton}=[font=\tiny,circle, fill=buttongray, text=gray, minimum height=1.2cm,minimum width=1.2cm]


\begin{titlepage}
  \begin{center}
    \vfil
    \includegraphics[width=12cm]{../nesizer_black.png}\\[3cm]

    \textsc{\LARGE{8-bit synthesizer}}\\[8cm]

    \textsc{\Large{Operating manual}}
  \end{center}
\end{titlepage}

\tableofcontents

%\maketitle

\chapter{User manual}

\section{Overview}

\section{The NESIZER}

\subsection{Sound channels}

At the heart of the \name is the NES APU chip, usually called 2A03 (or 2A07 if you use a chip from the PAL NES). The APU has five separate sound channels:

\begin{itemize}
\item \textbf{SQ1} and \textbf{SQ2}: These produce square waves with three selectable \emph{duty cycles} (pulse widths).
\item \textbf{TRI}: This channel produces triangular waves, but with a low 4 bit amplitude resolution. This results in the characteristic aliased NES bass and flute tones.
\item \textbf{NOISE}: This channel produces various forms of noise. The noise can be white noise, or pitched noise if the \lbl{LOOP} mode is engaged.
\item \textbf{DMC}: This channel can output 7-bit samples at a quick rate. The sampling rate is 16 kHz in the \name.
\end{itemize}

\subsection{Modulation}

Because the 2A03 is put under much tighter control in the \name than in a NES or Famicom console, the \name features extensive modulation capabilities.

\begin{itemize}
\item Three separate low frequency oscillators \textbf{LFO1}, \textbf{LFO2}, \textbf{LFO3} with selectable waveforms (ramp up, ramp down, sine wave, triangle wave or square wave)
\item Dedicated ADSR envelope generators for the square and noise channels
\item Portamento / glide for the square and triangle channels
\end{itemize}

\section{PROGRAM and PAGE 2 modes}

This is the active mode when the \name starts up. When in other modes, press the \btn{PROGRAM} button to switch to this mode.

In the programming mode, the various parameters of the sound channels and LFOs can be changed. A set of channel and LFO settings are collectively known as a \emph{patch}. The \name can save patches to memory, and has room for 100 patches.

\begin{tikzpicture}[align=center, node distance=1.5cm, auto]
  \node [mhbutton] (1) {SQ1};
  \node [mhbutton, right of=1] (2) {SQ2};
  \node [mhbutton, right of=2] (3) {TRI};
  \node [mhbutton, right of=3] (4) {NOISE};
  \node [mhbutton, right of=4] (5) {DMC};
  \node [hbutton, right of=5] (6) {LFO1};
  \node [hbutton, right of=6] (7) {LFO2};
  \node [hbutton, right of=7] (8) {LFO3};

  \node [hbutton, below=1] (9) {};
  \node [hbutton, right of=9] (10) {};
  \node [hbutton, right of=10] (11) {};
  \node [hbutton, right of=11] (12) {};
  \node [hbutton, right of=12] (13) {};
  \node [hbutton, right of=13] (14) {};
  \node [hbutton, right of=14] (15) {};
  \node [hbutton, right of=15] (16) {};
\end{tikzpicture}

\begin{tikzpicture}[align=center, node distance=1.5cm, auto]
  \node [dummybutton] (1) {};
  \node [dummybutton, right of=1] (2) {};
  \node [hcircbutton, right of=2] (3) {PROGRAM};
  \node [circbutton, right of=3] (4) {PAGE 2};
  \node [circbutton, right of=4] (5) {ASSIGNER};
  \node [circbutton, right of=5] (6) {SETTINGS};
  \node [dummybutton, right of=6] (7) {};
  \node [dummybutton, right of=7] (8) {};

  \node [hcircbutton, below=1] (8) {CLEAR};
  \node [hcircbutton, right of=8] (9) {SAVE};
  \node [dummybutton, right of=9] (10) {};
  \node [dummybutton, right of=10] (11) {};
  \node [dummybutton, right of=11] (12) {};
  \node [dummybutton, right of=12] (13) {};
  \node [hcircbutton, right of=13] (14) {UP};
  \node [hcircbutton, right of=14] (15) {DOWN};
\end{tikzpicture}

\subsection{Loading patches}
To load a patch, press either \btn{UP} or \btn{DOWN}. To go quickly up or down, press and hold the respective button. The current patch number is shown on the numeric display.

\subsection{Saving patches}
To save a patch, press \btn{SAVE}. The button will start to blink to indicate that you can select where to store the new patch. Use \btn{UP} and \btn{DOWN} to select where to store the patch. Press \btn{SAVE} again to store the patch at the selected location.

\emph{Note: When channel and LFO settings are changed, these are not saved until you press \btn{SAVE}.}

\subsection{Enabling and disabling channels}
To enable or disable a channel, press the corresponding channel button. When a channel is disabled, it does not produce sound when being triggered by the sequencer or incoming MIDI data.

\subsection{PAGE 2}

By pressing \btn{PAGE 2}, a second ``page'' og parameters become available. Some of them are channel parameters, and others are related to note assignment. When Page 2 is selected, the \name is still in programmer mode. The parameters that are avilable in page 2 are shown in yellow color in the figures below.

\subsection{Changing channel parameters}
To change a channel's parameter, press and hold the desired channel's button, and the desired parameter button. For example, to change the attack of the square 1 channel, press \btn{SQ1} and \btn{A}. The button LEDs will start to blink to indicate which channel parameter is being changed. Use the \btn{UP} and \btn{DOWN} buttons to change the parameter value. When you have the desired value, press \btn{SAVE}.

\subsubsection{Square channels}

\begin{tikzpicture}[align=center, node distance=1.5cm and 0.5cm]
  \node [mhbutton] (1) {SQ1};
  \node [mhbutton, right of=1] (2) {SQ2};
  \node [mbutton, right of=2] (3) {TRI};
  \node [mbutton, right of=3] (4) {NOISE};
  \node [mbutton, right of=4] (5) {DMC};
  \node [hbutton, right of=5] (6) {LFO1\\[4]\textcolor{yellow}{BEND}};
  \node [hbutton, right of=6] (7) {LFO2\\[4]\textcolor{yellow}{ENV MOD}};
  \node [hbutton, right of=7] (8) {LFO3\\[4]\textcolor{yellow}{VOL MOD}};

  \node [hbutton, below=1] (9) {DUTY};
  \node [hbutton, right of=9] (10) {DETUNE};
  \node [hbutton, right of=10] (11) {OCTAVE};
  \node [hbutton, right of=11] (12) {GLIDE};
  \node [hbutton, right of=12] (13) {A};
  \node [hbutton, right of=13] (14) {D};
  \node [hbutton, right of=14] (15) {S};
  \node [hbutton, right of=15] (16) {R};
\end{tikzpicture}

\begin{tabular}{l | l | l}
  Parameter & Description & Range\\ \hline
  \textbf{LFO1} & Intensity of modulation by LFO1 & 0 - 99\\
  \textbf{LFO2} & Intensity of modulation by LFO2 & 0 - 99\\
  \textbf{LFO3} & Intensity of modulation by LFO3 & 0 - 99\\
  \textbf{DUTY} & Duty cycle & 0 - 3\\
  \textbf{DETUNE} & Detuning & -9 - 9\\
  \textbf{OCTAVE} & Octave & -4 - 4\\
  \textbf{GLIDE} & Portamento glide time & 0 - 99\\
  \textbf{A} & Volume envelope attack & 0 - 99\\
  \textbf{D} & Volume envelope decay & 0 - 99\\
  \textbf{S} & Volume envelope sustain & 0 - 15\\
  \textbf{R} & Volume envelope release & 0 - 99\\
  \textbf{BEND} (page 2) & Bend wheel intensity in semitones & 0 - 24\\
  \textbf{ENV MOD} (page 2) & Pitch envelope modulation amount & -9 - 9\\
  \textbf{VOL MOD} (page 2) & Volume modulation by LFO3 & 0 - 16\\
\end{tabular}

\subsubsection{Triangle channel}

\begin{tikzpicture}[align=center, node distance=1.5cm and 0.5cm]
  \node [mbutton] (1) {SQ1};
  \node [mbutton, right of=1] (2) {SQ2};
  \node [mhbutton, right of=2] (3) {TRI};
  \node [mbutton, right of=3] (4) {NOISE};
  \node [mbutton, right of=4] (5) {DMC};
  \node [hbutton, right of=5] (6) {LFO1\\[4]\textcolor{yellow}{BEND}};
  \node [hbutton, right of=6] (7) {LFO2\\[4]\textcolor{yellow}{ENV MOD}};
  \node [hbutton, right of=7] (8) {LFO3};

  \node [button, below=1] (9) {};
  \node [hbutton, right of=9] (10) {DETUNE};
  \node [hbutton, right of=10] (11) {OCTAVE};
  \node [hbutton, right of=11] (12) {GLIDE};
  \node [button, right of=12] (13) {};
  \node [button, right of=13] (14) {};
  \node [button, right of=14] (15) {};
  \node [button, right of=15] (16) {};
\end{tikzpicture}

\begin{tabular}{l | l | l}
  Parameter & Description & Range\\ \hline
  \textbf{LFO1} & Intensity of modulation by LFO1 & 0 - 99\\
  \textbf{LFO2} & Intensity of modulation by LFO2 & 0 - 99\\
  \textbf{LFO3} & Intensity of modulation by LFO3 & 0 - 99\\
  \textbf{DETUNE} & Detuning & -9 - 9\\
  \textbf{OCTAVE} & Octave & -4 - 4\\
  \textbf{GLIDE} & Portamento glide time & 0 - 99\\
  \textbf{BEND} (page 2) & Bend wheel intensity in semitones & 0 - 24\\
  \textbf{ENV MOD} (page 2) & Pitch envelope modulation (noise envelope) & -9 - 9\\
\end{tabular}

\subsubsection{Noise channel}

\begin{tikzpicture}[align=center, node distance=1.5cm and 0.5cm]
  \node [mbutton] (1) {SQ1};
  \node [mbutton, right of=1] (2) {SQ2};
  \node [mbutton, right of=2] (3) {TRI};
  \node [mhbutton, right of=3] (4) {NOISE};
  \node [mbutton, right of=4] (5) {DMC};
  \node [hbutton, right of=5] (6) {LFO1\\[4]\textcolor{yellow}{BEND}};
  \node [hbutton, right of=6] (7) {LFO2\\[4]\textcolor{yellow}{ENV MOD}};
  \node [hbutton, right of=7] (8) {LFO3\\[4]\textcolor{yellow}{VOL MOD}};

  \node [hbutton, below=1] (9) {LOOP};
  \node [button, right of=9] (10) {};
  \node [button, right of=10] (11) {};
  \node [button, right of=11] (12) {};
  \node [hbutton, right of=12] (13) {A};
  \node [hbutton, right of=13] (14) {D};
  \node [hbutton, right of=14] (15) {S};
  \node [hbutton, right of=15] (16) {R};
\end{tikzpicture}

\begin{tabular}{l | l | l}
  Parameter & Description & Range\\ \hline
  \textbf{LFO1} & Intensity of modulation by LFO1 & 0 - 99\\
  \textbf{LFO2} & Intensity of modulation by LFO2 & 0 - 99\\
  \textbf{LFO3} & Intensity of modulation by LFO3 & 0 - 99\\
  \textbf{LOOP} & Looped noise & on/off\\
  \textbf{A} & Volume envelope attack & 0 - 99\\
  \textbf{D} & Volume envelope decay & 0 - 99\\
  \textbf{S} & Volume envelope sustain & 0 - 15\\
  \textbf{R} & Volume envelope release & 0 - 99\\
  \textbf{BEND} (page 2) & Bend wheel intensity (in steps) & 0 - 15\\
  \textbf{ENV MOD} (page 2) & Pitch envelope modulation amount & -9 - 9\\
  \textbf{VOL MOD} (page 2) & Volume modulation by LFO3 & 0 - 16\\
\end{tabular}

\subsubsection{DMC channel}

\begin{tikzpicture}[align=center, node distance=1.5cm and 0.5cm]
  \node [mbutton] (1) {SQ1};
  \node [mbutton, right of=1] (2) {SQ2};
  \node [mbutton, right of=2] (3) {TRI};
  \node [mbutton, right of=3] (4) {NOISE};
  \node [mhbutton, right of=4] (5) {DMC};
  \node [button, right of=5] (6) {LFO1};
  \node [button, right of=6] (7) {LFO2};
  \node [button, right of=7] (8) {LFO3};

  \node [hbutton, below=1] (9) {LOOP};
  \node [hbutton, right of=9] (10) {DIVIDER};
  \node [button, right of=10] (11) {};
  \node [button, right of=11] (12) {};
  \node [button, right of=12] (13) {};
  \node [button, right of=13] (14) {};
  \node [button, right of=14] (15) {};
  \node [button, right of=15] (16) {};
\end{tikzpicture}

\begin{tabular}{l | l | l}
  Parameter & Description & Range\\ \hline
  \textbf{LOOP} & Play samples in loop & on/off\\
  %\textbf{DIVIDER} & Divider for sample rate & 1 - 4\\
\end{tabular}

\subsubsection{LFOs}

\begin{tikzpicture}[align=center, node distance=1.5cm and 0.5cm]
  \node [mbutton] (1) {SQ1};
  \node [mbutton, right of=1] (2) {SQ2};
  \node [mbutton, right of=2] (3) {TRI};
  \node [mbutton, right of=3] (4) {NOISE};
  \node [mbutton, right of=4] (5) {DMC};
  \node [hbutton, right of=5] (6) {LFO1};
  \node [hbutton, right of=6] (7) {LFO2};
  \node [hbutton, right of=7] (8) {LFO3};

  \node [hbutton, below=1] (9) {WAVE};
  \node [button, right of=9] (10) {};
  \node [button, right of=10] (11) {};
  \node [button, right of=11] (12) {};
  \node [button, right of=12] (13) {};
  \node [button, right of=13] (14) {};
  \node [button, right of=14] (15) {};
  \node [button, right of=15] (16) {};
\end{tikzpicture}

\begin{tabular}{l | l | l}
  Parameter & Description & Range\\ \hline
  \textbf{WAVE} & LFO waveform & 1 - sine\\
  & & 2 - ramp up\\
  & & 3 - ramp down\\
  & & 4 - square\\
  & & 5 - triangle
\end{tabular}


\section{SETTINGS}

When in the settings mode, various aspects of how the \name operates can be changed. In this mode, the buttons have the following functions:

\begin{tikzpicture}[align=center, node distance=1.5cm]
  \node [mhbutton] (1) {SQ1};
  \node [mhbutton, right of=1] (2) {SQ2};
  \node [mhbutton, right of=2] (3) {TRI};
  \node [mhbutton, right of=3] (4) {NOISE};
  \node [mhbutton, right of=4] (5) {DMC};
  \node [hbutton, right of=5] (6) {MIDI CHN};
  \node [hbutton, right of=6] (7) {BATTERY\\VOLTAGE};
  \node [hbutton, right of=7] (8) {CHIP TYPE};

  \node [hbutton, below=1] (9) {PATCH\\RESET};
  \node [button, right of=9] (10) {};
  \node [button, right of=10] (11) {};
  \node [button, right of=11] (12) {};
  \node [button, right of=12] (13) {};
  \node [button, right of=13] (14) {};
  \node [hbutton, right of=14] (15) {SAMPLE\\RESET};
  \node [hbutton, right of=15] (16) {SAMPLE\\DELETE};

  \node [circbutton, below= of 9] (17) {};
  \node [circbutton, right of=17] (18) {};
  \node [dummybutton, right of=18] (19) {};
  \node [dummybutton, right of=19] (20) {};
  \node [dummybutton, right of=20] (21) {};
  \node [dummybutton, right of=21] (22) {};
  \node [hcircbutton, right of=22] (23) {UP};
  \node [hcircbutton, right of=23] (24) {DOWN};
\end{tikzpicture}

\btn{UP} and \btn{DOWN} can be used to select a sample number. Sample numbers that are occupied are marked with a dot on the display. A selected sample can be deleted by pressing \btn{DELETE}.

\subsection{MIDI}

The \name can be controlled externally using MIDI.

\subsubsection{Assigning MIDI channels}

Each of the five sound channels can be assigned to any of the 16 MIDI channels, and will then only respond to incoming messages on the selected channel.

To select a MIDI channel, enter the \lbl{SETTINGS} mode by pressing \btn{SETTINGS}. Hold down \btn{MIDI CHANNEL} and then press the desired channel's button. The LEDs will flash and you can select one of the 16 MIDI channels using the \btn{UP} and \btn{DOWN} buttons. If you select the value 0, the sound channel will not listen to any MIDI channel.

\subsection{Checking the battery voltage}
The NESIZER uses a battery for keeping the RAM storing the patches and samples alive when main power is disconnected. To check the battery's voltage, press and hold \btn{BATTERY VOLTAGE}. As long as the button is pressed, the battery's voltage will be shown in the display. If the battery voltage is below 2.6 V, it should be replaced. On startup, the NESIZER will give a warning if the battery is 2.5 V or less. The display will flash \verb+bL+ (Battery Low) for a short duration.

\subsection{Checking the 2A03 type}
To see which 2A03 chip is being used, press the LFO3 button. One of the following numbers will show up:
\begin{itemize}
\item 12: Genuine RP2A03
\item 15: 2A03 clone, e.g. 6527P
\item 16: 2A07 (PAL version of 2A03)
\end{itemize}

\subsection{Resetting patches}

Press \btn{PATCH RESET} to delete all patches. Every patch will be initialized to a basic patch with no channels enabled, square duty cycles set at 50\% and full envelope sustain levels with no attack, decay or release.

\subsection{Maintaining samples}

When in SETTINGS mode, the up and down buttons are used to select DMC samples. A dot appearing on the display indicates that the selected sample location is occupied. When an occupied sample is selected, it can be deleted by pressing \btn{SAMPLE DELETE}. All samples can be erased by pressing \btn{SAMPLE RESET}.


\end{document}
