\documentclass[a4paper, 12p]{extarticle}
\usepackage[centering, margin=3.0cm]{geometry}
\usepackage{cmbright}
\usepackage{graphicx}
\usepackage[intlimits]{amsmath}
\usepackage{amssymb}
\usepackage{amsthm}
\usepackage{parskip}
\usepackage{pgf}
\usepackage{tikz}
\usetikzlibrary{shapes}
\usetikzlibrary{positioning}
\usetikzlibrary{calc}
\usepackage{enumerate}
\usepackage{framed}
\usepackage{tikz}

\setcounter{secnumdepth}{0}

\newcommand \name {NESIZER }
\newcommand {\btn}[1] {\framebox{\textbf{\footnotesize #1}}}
\newcommand {\cbtn}[1] {\tikz \node[draw,circle]{\textbf{\footnotesize #1}};}
\newcommand {\lbl}[1] {\emph{\footnotesize #1}}
\newcommand \dsp \texttt

\AddToHook{cmd/section/before}{\clearpage}

\title{NESIZER}
\author{}
\date{}


\begin{document}

\definecolor{buttonredh}{RGB}{255,0,0}
\definecolor{buttonred}{RGB}{240, 161, 161}
\definecolor{buttongray}{RGB}{71,71,71}

\tikzstyle{mbutton}=[font=\tiny,rectangle, draw, fill=buttonred, text=gray, minimum height=1.2cm, minimum width=1.2cm]
\tikzstyle{mhbutton}=[font=\tiny,rectangle, draw, fill=buttonredh, minimum height=1.2cm,minimum width=1.2cm]

\tikzstyle{hbutton}=[font=\tiny,rectangle, draw, fill=black, text=white, minimum height=1.2cm, minimum width=1.2cm]
\tikzstyle{button}=[font=\tiny,rectangle, draw, fill=buttongray, text=gray, minimum height=1.2cm,minimum width=1.2cm]

\tikzstyle{dummybutton}=[font=\tiny,rectangle, fill=white, text=white, minimum height=1cm,minimum width=1cm]
\tikzstyle{hcircbutton}=[font=\tiny,circle, fill=black, text=white, minimum height=1.2cm,minimum width=1.2cm]
\tikzstyle{circbutton}=[font=\tiny,circle, fill=buttongray, text=gray, minimum height=1.2cm,minimum width=1.2cm]
\tikzstyle{selcircbutton}=[font=\tiny,circle,line width=0.7mm,draw=green!80, fill=black, text=white, minimum height=1.2cm,minimum width=1.2cm]

\begin{titlepage}
  \begin{center}
    \vfil
    \includegraphics[width=12cm]{../nesizer_black.png}\\[3cm]

    \textsc{\LARGE{8-bit synthesizer}}\\[8cm]

    \textsc{\Large{Operating manual}}
  \end{center}
\end{titlepage}

\tableofcontents

%\maketitle

\section{The NESIZER}

\subsection{Sound channels}

At the heart of the \name is the NES APU chip, usually called 2A03 (or 2A07 if you use a chip from the PAL NES). The APU has five separate sound channels:

\begin{itemize}
\item \textbf{SQ1} and \textbf{SQ2}: These produce square waves with three selectable \emph{duty cycles} (pulse widths).
\item \textbf{TRI}: This channel produces triangular waves, but with a low 4 bit amplitude resolution. This results in the characteristic aliased NES bass and flute tones.
\item \textbf{NOISE}: This channel produces various forms of noise. The noise can be white noise, or pitched noise if the \lbl{LOOP} mode is engaged.
\item \textbf{DMC}: This channel can output 7-bit samples at a quick rate. The sampling rate is 16 kHz in the \name.
\end{itemize}

\subsection{Modulation}

Because the 2A03 is put under much tighter control in the \name than in a NES or Famicom console, the \name features extensive modulation capabilities.

\begin{itemize}
\item Three separate low frequency oscillators \textbf{LFO1}, \textbf{LFO2}, \textbf{LFO3} with selectable waveforms (ramp up, ramp down, sine wave, triangle wave or square wave)
\item Dedicated ADSR envelope generators for the square and noise channels
\item Portamento / glide for the square and triangle channels
\end{itemize}

\section{PROGRAM and PAGE 2 modes}

This is the active mode when the \name starts up. When in other modes, press the \btn{PROGRAM} button to switch to this mode.

In the programming mode, the various parameters of the sound channels and LFOs can be changed. A set of channel and LFO settings are collectively known as a \emph{patch}. The \name can save patches to memory, and has room for 100 patches.

\begin{tikzpicture}[align=center, node distance=1.5cm, auto]
  \node(scope1) {
    \begin{tikzpicture}
      \node [mhbutton] (1) {SQ1};
      \node [mhbutton, right of=1] (2) {SQ2};
      \node [mhbutton, right of=2] (3) {TRI};
      \node [mhbutton, right of=3] (4) {NOISE};
      \node [mhbutton, right of=4] (5) {DMC};
      \node [hbutton, right of=5] (6) {LFO1};
      \node [hbutton, right of=6] (7) {LFO2};
      \node [hbutton, right of=7] (8) {LFO3};

      \node [hbutton, below=1] (9) {};
      \node [hbutton, right of=9] (10) {};
      \node [hbutton, right of=10] (11) {};
      \node [hbutton, right of=11] (12) {};
      \node [hbutton, right of=12] (13) {};
      \node [hbutton, right of=13] (14) {};
      \node [hbutton, right of=14] (15) {};
      \node [hbutton, right of=15] (16) {};
    \end{tikzpicture}
  };

  \node[at={($(scope1.south)$)},anchor=north] (scope2) {
    \begin{tikzpicture}
      \node [dummybutton] (1) {};
      \node [dummybutton, right of=1] (2) {};
      \node [selcircbutton, right of=2] (3) {PROGRAM};
      \node [selcircbutton, right of=3] (4) {PAGE 2};
      \node [circbutton, right of=4] (5) {SEQ.};
      \node [circbutton, right of=5] (6) {SETTINGS};
      \node [dummybutton, right of=6] (7) {};
      \node [dummybutton, right of=7] (8) {};

      \node [hcircbutton, below=0.1cm of 1] (9) {CLEAR};
      \node [hcircbutton, right of=9] (10) {SAVE};
      \node [dummybutton, right of=10] (11) {};
      \node [dummybutton, right of=11] (12) {};
      \node [dummybutton, right of=12] (13) {};
      \node [dummybutton, right of=13] (14) {};
      \node [hcircbutton, right of=14] (15) {UP};
      \node [hcircbutton, right of=15] (16) {DOWN};
    \end{tikzpicture}
  };
\end{tikzpicture}

\subsection{Loading patches}
To load a patch, press either \btn{UP} or \btn{DOWN}. To go quickly up or down, press and hold the respective button. The current patch number is shown on the numeric display.

\subsection{Saving patches}
To save a patch, press \btn{SAVE}. The button will start to blink to indicate that you can select where to store the new patch. Use \btn{UP} and \btn{DOWN} to select where to store the patch. Press \btn{SAVE} again to store the patch at the selected location.

\emph{Note: When channel and LFO settings are changed, these are not saved until you press \btn{SAVE}.}

\subsection{Enabling and disabling channels}
To enable or disable a channel, press the corresponding channel button. When a channel is disabled, it does not produce sound when being triggered by the sequencer or incoming MIDI data.

\subsection{PAGE 2}

By pressing \btn{PAGE 2}, a second ``page'' og parameters become available. Some of them are channel parameters, and others are related to note assignment. When Page 2 is selected, the \name is still in programming mode, so pressing \btn{UP} or \btn{DOWN} will change the current patch, and will discard any changes that haven't been saved. The parameters that are avilable in page 2 are shown in yellow color in the figures below.

\subsection{Changing channel parameters}
To change a channel's parameter, press and hold the desired channel's button, and the desired parameter button. For example, to change the attack of the square 1 channel, press \btn{SQ1} and \btn{A}. The button LEDs will start to blink to indicate which channel parameter is being changed. Use the \btn{UP} and \btn{DOWN} buttons to change the parameter value. When you have the desired value, press \btn{SAVE}.

\subsubsection{Square channels}

\begin{tikzpicture}[align=center, node distance=1.5cm and 0.5cm]
  \node [mhbutton] (1) {SQ1};
  \node [mhbutton, right of=1] (2) {SQ2};
  \node [mbutton, right of=2] (3) {TRI};
  \node [mbutton, right of=3] (4) {NOISE};
  \node [mbutton, right of=4] (5) {DMC};
  \node [hbutton, right of=5] (6) {LFO1\\[4]\textcolor{yellow}{BEND}};
  \node [hbutton, right of=6] (7) {LFO2\\[4]\textcolor{yellow}{ENV MOD}};
  \node [hbutton, right of=7] (8) {LFO3\\[4]\textcolor{yellow}{VOL MOD}};

  \node [hbutton, below=1] (9) {DUTY\\[4]\textcolor{yellow}{SPLIT HALF}};
  \node [hbutton, right of=9] (10) {DETUNE};
  \node [hbutton, right of=10] (11) {OCTAVE};
  \node [hbutton, right of=11] (12) {GLIDE};
  \node [hbutton, right of=12] (13) {A};
  \node [hbutton, right of=13] (14) {D};
  \node [hbutton, right of=14] (15) {S};
  \node [hbutton, right of=15] (16) {R};
\end{tikzpicture}

\begin{tabular}{l | l | l}
  Parameter & Description & Range\\ \hline
  \textbf{LFO1} & Intensity of modulation by LFO1 & 0 - 99\\
  \textbf{LFO2} & Intensity of modulation by LFO2 & 0 - 99\\
  \textbf{LFO3} & Intensity of modulation by LFO3 & 0 - 99\\
  \textbf{DUTY} & Duty cycle & 0 - 3\\
  \textbf{DETUNE} & Detuning & -9 - 9\\
  \textbf{OCTAVE} & Octave & -4 - 4\\
  \textbf{GLIDE} & Portamento glide time & 0 - 99\\
  \textbf{A} & Volume envelope attack & 0 - 99\\
  \textbf{D} & Volume envelope decay & 0 - 99\\
  \textbf{S} & Volume envelope sustain & 0 - 15\\
  \textbf{R} & Volume envelope release & 0 - 99\\
  \textbf{BEND} (page 2) & Bend wheel intensity in semitones & 0 - 24\\
  \textbf{ENV MOD} (page 2) & Pitch envelope modulation amount & -9 - 9\\
  \textbf{VOL MOD} (page 2) & Volume modulation by LFO3 & 0 - 16\\
  \textbf{SPLIT HALF} (page 2) & See section 1.3.6 & Lower - Upper\\
\end{tabular}

\subsubsection{Triangle channel}

\begin{tikzpicture}[align=center, node distance=1.5cm and 0.5cm]
  \node [mbutton] (1) {SQ1};
  \node [mbutton, right of=1] (2) {SQ2};
  \node [mhbutton, right of=2] (3) {TRI};
  \node [mbutton, right of=3] (4) {NOISE};
  \node [mbutton, right of=4] (5) {DMC};
  \node [hbutton, right of=5] (6) {LFO1\\[4]\textcolor{yellow}{BEND}};
  \node [hbutton, right of=6] (7) {LFO2\\[4]\textcolor{yellow}{ENV MOD}};
  \node [hbutton, right of=7] (8) {LFO3};

  \node [hbutton, below=1] (9) {\textcolor{yellow}{SPLIT HALF}};
  \node [hbutton, right of=9] (10) {DETUNE};
  \node [hbutton, right of=10] (11) {OCTAVE};
  \node [hbutton, right of=11] (12) {GLIDE};
  \node [button, right of=12] (13) {};
  \node [button, right of=13] (14) {};
  \node [button, right of=14] (15) {};
  \node [button, right of=15] (16) {};
\end{tikzpicture}

\begin{tabular}{l | l | l}
  Parameter & Description & Range\\ \hline
  \textbf{LFO1} & Intensity of modulation by LFO1 & 0 - 99\\
  \textbf{LFO2} & Intensity of modulation by LFO2 & 0 - 99\\
  \textbf{LFO3} & Intensity of modulation by LFO3 & 0 - 99\\
  \textbf{DETUNE} & Detuning & -9 - 9\\
  \textbf{OCTAVE} & Octave & -4 - 4\\
  \textbf{GLIDE} & Portamento glide time & 0 - 99\\
  \textbf{BEND} (page 2) & Bend wheel intensity in semitones & 0 - 24\\
  \textbf{ENV MOD} (page 2) & Pitch envelope modulation (noise envelope) & -9 - 9\\
  \textbf{SPLIT HALF} (page 2) & See section 1.3.6 & Lower - Upper\\
\end{tabular}

\subsubsection{Noise channel}

\begin{tikzpicture}[align=center, node distance=1.5cm and 0.5cm]
  \node [mbutton] (1) {SQ1};
  \node [mbutton, right of=1] (2) {SQ2};
  \node [mbutton, right of=2] (3) {TRI};
  \node [mhbutton, right of=3] (4) {NOISE};
  \node [mbutton, right of=4] (5) {DMC};
  \node [hbutton, right of=5] (6) {LFO1\\[4]\textcolor{yellow}{BEND}};
  \node [hbutton, right of=6] (7) {LFO2\\[4]\textcolor{yellow}{ENV MOD}};
  \node [hbutton, right of=7] (8) {LFO3\\[4]\textcolor{yellow}{VOL MOD}};

  \node [hbutton, below=1] (9) {LOOP\\[4]\textcolor{yellow}{SPLIT HALF}};
  \node [button, right of=9] (10) {};
  \node [button, right of=10] (11) {};
  \node [button, right of=11] (12) {};
  \node [hbutton, right of=12] (13) {A};
  \node [hbutton, right of=13] (14) {D};
  \node [hbutton, right of=14] (15) {S};
  \node [hbutton, right of=15] (16) {R};
\end{tikzpicture}

\begin{tabular}{l | l | l}
  Parameter & Description & Range\\ \hline
  \textbf{LFO1} & Intensity of modulation by LFO1 & 0 - 99\\
  \textbf{LFO2} & Intensity of modulation by LFO2 & 0 - 99\\
  \textbf{LFO3} & Intensity of modulation by LFO3 & 0 - 99\\
  \textbf{LOOP} & Looped noise & on/off\\
  \textbf{A} & Volume envelope attack & 0 - 99\\
  \textbf{D} & Volume envelope decay & 0 - 99\\
  \textbf{S} & Volume envelope sustain & 0 - 15\\
  \textbf{R} & Volume envelope release & 0 - 99\\
  \textbf{BEND} (page 2) & Bend wheel intensity (in steps) & 0 - 15\\
  \textbf{ENV MOD} (page 2) & Pitch envelope modulation amount & -9 - 9\\
  \textbf{VOL MOD} (page 2) & Volume modulation by LFO3 & 0 - 16\\
  \textbf{SPLIT HALF} (page 2) & See section 1.3.6 & Lower - Upper\\
\end{tabular}

\subsubsection{DMC channel}

\begin{tikzpicture}[align=center, node distance=1.5cm and 0.5cm]
  \node [mbutton] (1) {SQ1};
  \node [mbutton, right of=1] (2) {SQ2};
  \node [mbutton, right of=2] (3) {TRI};
  \node [mbutton, right of=3] (4) {NOISE};
  \node [mhbutton, right of=4] (5) {DMC};
  \node [button, right of=5] (6) {LFO1};
  \node [button, right of=6] (7) {LFO2};
  \node [button, right of=7] (8) {LFO3};

  \node [hbutton, below=1] (9) {LOOP};
  \node [hbutton, right of=9] (10) {};
  \node [button, right of=10] (11) {};
  \node [button, right of=11] (12) {};
  \node [button, right of=12] (13) {};
  \node [button, right of=13] (14) {};
  \node [button, right of=14] (15) {};
  \node [button, right of=15] (16) {};
\end{tikzpicture}

\begin{tabular}{l | l | l}
  Parameter & Description & Range\\ \hline
  \textbf{LOOP} & Play samples in loop & on/off\\
  %\textbf{DIVIDER} & Divider for sample rate & 1 - 4\\
\end{tabular}

\subsubsection{LFOs}

\begin{tikzpicture}[align=center, node distance=1.5cm and 0.5cm]
  \node [mbutton] (1) {SQ1};
  \node [mbutton, right of=1] (2) {SQ2};
  \node [mbutton, right of=2] (3) {TRI};
  \node [mbutton, right of=3] (4) {NOISE};
  \node [mbutton, right of=4] (5) {DMC};
  \node [hbutton, right of=5] (6) {LFO1};
  \node [hbutton, right of=6] (7) {LFO2};
  \node [hbutton, right of=7] (8) {LFO3};

  \node [hbutton, below=1] (9) {WAVE};
  \node [button, right of=9] (10) {};
  \node [button, right of=10] (11) {};
  \node [button, right of=11] (12) {};
  \node [button, right of=12] (13) {};
  \node [button, right of=13] (14) {};
  \node [button, right of=14] (15) {};
  \node [button, right of=15] (16) {};
\end{tikzpicture}

\begin{tabular}{l | l | l}
  Parameter & Description & Range\\ \hline
  \textbf{WAVE} & LFO waveform & 1 - sine\\
  & & 2 - ramp up\\
  & & 3 - ramp down\\
  & & 4 - square\\
  & & 5 - triangle
\end{tabular}

\subsection{Note assignment}

\begin{tikzpicture}[align=center, node distance=1.5cm and 0.5cm]
  \node(scope1) {
    \begin{tikzpicture}
      \node [mhbutton] (1) {SQ1};
      \node [mhbutton, right of=1] (2) {SQ2};
      \node [mhbutton, right of=2] (3) {TRI};
      \node [mhbutton, right of=3] (4) {NOISE};
      \node [mhbutton, right of=4] (5) {DMC};
      \node [button, right of=5] (6) {LFO1};
      \node [button, right of=6] (7) {LFO2};
      \node [button, right of=7] (8) {LFO3};

      \node [hbutton, below=1] (9) {\textcolor{yellow}{SPLIT HALF}};
      \node [hbutton, right of=9] (10) {\textcolor{yellow}{SPLIT}\\\textcolor{yellow}{ON/OFF}};
      \node [hbutton, right of=10] (11) {\textcolor{yellow}{SPLIT}\\\textcolor{yellow}{SET POINT}};
      \node [button, right of=11] (12) {};
      \node [hbutton, right of=12] (13) {\textcolor{yellow}{LOWER}\\\textcolor{yellow}{POLY}};
      \node [hbutton, right of=13] (14) {\textcolor{yellow}{LOWER}\\\textcolor{yellow}{MONO}};
      \node [hbutton, right of=14] (15) {\textcolor{yellow}{(UPPER)}\\\textcolor{yellow}{POLY}};
      \node [hbutton, right of=15] (16) {\textcolor{yellow}{(UPPER)}\\\textcolor{yellow}{MONO}};
    \end{tikzpicture}
  };

  \node[at={($(scope1.south)$)},anchor=north] (scope2) {
    \begin{tikzpicture}
      \node [dummybutton] (1) {};
      \node [dummybutton, right of=1] (2) {};
      \node [circbutton, right of=2] (3) {PROGRAM};
      \node [selcircbutton, right of=3] (4) {PAGE 2};
      \node [circbutton, right of=4] (5) {SEQ.};
      \node [circbutton, right of=5] (6) {SETTINGS};
      \node [dummybutton, right of=6] (7) {};
      \node [dummybutton, right of=7] (8) {};

      \node [hcircbutton, below=0.1cm of 1] (8) {CLEAR};
      \node [hcircbutton, right of=8] (9) {SAVE};
      \node [dummybutton, right of=9] (10) {};
      \node [dummybutton, right of=10] (11) {};
      \node [dummybutton, right of=11] (12) {};
      \node [dummybutton, right of=12] (13) {};
      \node [hcircbutton, right of=13] (14) {UP};
      \node [hcircbutton, right of=14] (15) {DOWN};
    \end{tikzpicture}
  };
\end{tikzpicture}

\subsubsection{Monophonic and polyphonic modes}

When several channels are sharing the same MIDI channel, notes can be assigned either monophonically or polyphonically. In \textbf{monophonic} mode, all channels will play the same incoming MIDI note, and any new note will cut off the previous. This mode is selected by pressing \btn{(UPPER) MONO}. In \textbf{polyphonic} mode, the channels will be allocated in turn to each of the incoming MIDI notes. This mode is selected by pressing \btn{(UPPER) POLY}.

\subsubsection{Splitting the keyboard}

The MIDI keyboard\footnote{Not necessarily a physical keyboard, but more generally the range of MIDI notes.} can be split into two sections, a \emph{lower} and a \emph{upper} section. To activate this split, press \btn{SPLIT ON/OFF}. When the split is active, the \emph{splitting point} sets the boundary between the lower and upper half. To change the splitting point, press \btn{SPLIT SET POINT}, and select the note where you want the split. The notes are shown with a letter in the first half of the display, and the octave number in the second. A dot on the first display indicates a flat note. For instance, \verb+E.5+ would indicate the note E$\flat$ in the fourth octave (as numbered in MIDI).

The channels can be assigned to either section by pressing and holding the corresponding channel button and then pressing \btn{SPLIT HALF}. The sections are named on the display as \verb+Lo+ and \verb+UP+, short for Lower and Upper, respectively. When a channel has been assigned to one of the halves, that channel will only play a note when that note is in the same half.

Several channels can be assigned to the same half of the keyboard, so there are separate monophonic and polypohnic settings for each half. Use \btn{LOWER POLY} and \btn{LOWER MONO} to select the assignment mode for the lower half. Similarly, use \btn{(UPPER) POLY} and \btn{(UPPER) MONO} to select the assignment mode for the upper half (these are the same buttons as used for assignment mode when split is off).


\section{SEQUENCER}

In the sequencer mode, 16-note patterns can be created and played back. The sequencer is somewhat more complicated to use than the programmer. The user interface employs several \emph{levels}.

While a pattern is playing, you can change the NESIZER into any of the other modes. For example, the current patch can be edited while the sequence is playing.

\subsection{Selection mode}

When first entering the sequencer, it is at the \emph{selection layer}. This is where you select a pattern for playback or editing, as well as adjusting tempo.

\begin{tikzpicture}[align=center, node distance=1.5cm, auto]
  \node(scope1) {
    \begin{tikzpicture}
      \node [mhbutton] (1) {SQ1};
      \node [mhbutton, right of=1] (2) {SQ2};
      \node [mhbutton, right of=2] (3) {TRI};
      \node [mhbutton, right of=3] (4) {NOISE};
      \node [mhbutton, right of=4] (5) {DMC};
      \node [hbutton, right of=5] (6) {MIDI OUT\\CHANNEL};
      \node [button, right of=6] (7) {};
      \node [button, right of=7] (8) {};

      \node [hbutton, below=1] (9) {RECORD};
      \node [button, right of=9] (10) {};
      \node [button, right of=10] (11) {};
      \node [button, right of=11] (12) {};
      \node [button, right of=12] (13) {};
      \node [button, right of=13] (14) {};
      \node [hbutton, right of=14] (15) {END POINT};
      \node [hbutton, right of=15] (16) {SCALE};
    \end{tikzpicture}
  };

  \node[at={($(scope1.south)$)},anchor=north] (scope2) {
    \begin{tikzpicture}
      \node [dummybutton] (1) {};
      \node [dummybutton, right of=1] (2) {};
      \node [circbutton, right of=2] (3) {PROGRAM};
      \node [circbutton, right of=3] (4) {PAGE 2};
      \node [selcircbutton, right of=4] (5) {SEQ.};
      \node [circbutton, right of=5] (6) {SETTINGS};
      \node [dummybutton, right of=6] (7) {};
      \node [dummybutton, right of=7] (8) {};

      \node [hcircbutton, below=0.1cm of 1] (8) {PLAY/\\STOP};
      \node [hcircbutton, right of=8] (9) {SAVE};
      \node [dummybutton, right of=9] (10) {};
      \node [dummybutton, right of=10] (11) {};
      \node [dummybutton, right of=11] (12) {};
      \node [dummybutton, right of=12] (13) {};
      \node [hcircbutton, right of=13] (14) {UP};
      \node [hcircbutton, right of=14] (15) {DOWN};
    \end{tikzpicture}
  };
\end{tikzpicture}

\subsubsection{Selecting a sequence}

To select a new pattern, first make sure you are at the top level and not already editing a pattern. If you are, close the current pattern by pressing \btn{BACK}. Use \btn{UP} and \btn{DOWN} to select the pattern you want to edit. To edit the selected pattern, press the button for the channel you want to edit. The sequencer will then enter editing mode.

\subsubsection{Playing back a sequence}

To play back a selected sequence, press \btn{PLAY}. The sequence will loop indefinitely. Press \btn{PLAY} again to stop it. While a pattern is playing, \btn{UP} and \btn{DOWN} can be used to adjust the tempo.

\subsubsection{Editing a sequence}

To edit a pattern in the selected sequence, press the channel button corresponding to the one you want to edit. The sequencer will then enter pattern editing mode.

\subsubsection{Saving a sequence}

Any changes you have done to a sequence will not be saved until you press \btn{SAVE}. Saving sequences is done exactly the same way as saving programs.

\subsubsection{Scaling}

By pressing \btn{SCALE}, you can set the \emph{scale} of the sequence. The scale determines whether the 16-step sequence should represent one, two or four bars. When the scale is 1, each step in the sequence represents a 16th note in the bar. If the scale is 2, each step represents an 8th note. And finally, if the scale is set to 4, each step represents a quarter note.

\subsubsection{Setting the end point}

Sequences have the full length of 16 notes by default. To create sequences that fit with other time signatures, press \btn{END POINT}. The 16 pattern position LEDs will start blinking. Press the pattern position button that you want to be the last one in the sequence. The LEDs will stop blinking, and the new end point has been set.

\subsubsection{Selecting MIDI output channels}
Each of the five channel patterns can be assigned to an output MIDI channel. To do this, press and hold \btn{MIDI OUT CHANNEL} while pressing the button for the desired channel to assign an output MIDI channel to. When playing any sequence, notes from that pattern will then play on the selected MIDI channel, while still playing on the corresponding 2A03 channel.

\emph{Tip: By using a patch where no 2A03 channels are enabled, the sequencer can be used as a general MIDI sequencer.}

\subsection{Pattern editing mode}

When in pattern editing mode, the buttons take on the following functions:

\begin{tikzpicture}[align=center, node distance=1.5cm, auto]
  \node(scope1) {
    \begin{tikzpicture}
      \node [mhbutton] (1) {1};
      \node [mhbutton, right of=1] (2) {2};
      \node [mhbutton, right of=2] (3) {3};
      \node [mhbutton, right of=3] (4) {4};
      \node [mhbutton, right of=4] (5) {5};
      \node [hbutton, right of=5] (6) {6};
      \node [hbutton, right of=6] (7) {7};
      \node [hbutton, right of=7] (8) {8};

      \node [hbutton, below=1] (9) {9};
      \node [hbutton, right of=9] (10) {10};
      \node [hbutton, right of=10] (11) {11};
      \node [hbutton, right of=11] (12) {12};
      \node [hbutton, right of=12] (13) {13};
      \node [hbutton, right of=13] (14) {14};
      \node [hbutton, right of=14] (15) {15};
      \node [hbutton, right of=15] (16) {16};
    \end{tikzpicture}
  };

  \node[at={($(scope1.south)$)},anchor=north] (scope2) {
    \begin{tikzpicture}
      \node [dummybutton] (1) {};
      \node [dummybutton, right of=1] (2) {};
      \node [circbutton, right of=2] (3) {PROGRAM};
      \node [circbutton, right of=3] (4) {PAGE 2};
      \node [selcircbutton, right of=4] (5) {SEQ.};
      \node [circbutton, right of=5] (6) {SETTINGS};
      \node [dummybutton, right of=6] (7) {};
      \node [dummybutton, right of=7] (8) {};

      \node [hcircbutton, below=0.1cm of 1] (9) {BACK};
      \node [circbutton, right of=9] (10) {};
      \node [dummybutton, right of=10] (11) {};
      \node [dummybutton, right of=11] (12) {};
      \node [dummybutton, right of=12] (13) {};
      \node [dummybutton, right of=13] (14) {};
      \node [circbutton, right of=14] (15) {UP};
      \node [circbutton, right of=15] (16) {DOWN};
    \end{tikzpicture}
  };
\end{tikzpicture}

To help you remember which channel's pattern you're currently editing, the display will indicate a number ranging from 1 to 5, corresponding to SQ1, SQ2, TRI, NOISE and DMC, respectively.

Each of the 16 top buttons represent a position in the pattern. To place a note at a particular position in the pattern, press the corresponding button. The sequencer will then enter note entering mode, indicated by the LED on the button flashing. The next section details how you enter a note.

After a note has been entered, the button's LED will light up to indicate that a note is present at that position in the pattern.

When you're done working on the channel's pattern, press \btn{BACK} to go back to pattern selection mode.

\subsection{Note entering mode}

When a note has been selected in a pattern, the LED for the corresponding button will flash. The buttons now take on the following functions:

\begin{tikzpicture}[align=center, node distance=1.5cm, auto]
  \node(scope1) {
    \begin{tikzpicture}
      \node [mhbutton] (1) {C\#};
      \node [mhbutton, right of=1] (2) {D\#};
      \node [mhbutton, right of=2] (3) {};
      \node [mhbutton, right of=3] (4) {F\#};
      \node [mhbutton, right of=4] (5) {G\#};
      \node [hbutton, right of=5] (6) {A\#};
      \node [hbutton, right of=6] (7) {OCTAVE};
      \node [hbutton, right of=7] (8) {ERASE};

      \node [hbutton, below=1] (9) {C};
      \node [hbutton, right of=9] (10) {D};
      \node [hbutton, right of=10] (11) {E};
      \node [hbutton, right of=11] (12) {F};
      \node [hbutton, right of=12] (13) {G};
      \node [hbutton, right of=13] (14) {A};
      \node [hbutton, right of=14] (15) {B};
      \node [hbutton, right of=15] (16) {C};
    \end{tikzpicture}
  };

  \node[at={($(scope1.south)$)},anchor=north] (scope2) {
    \begin{tikzpicture}
      \node [dummybutton] (1) {};
      \node [dummybutton, right of=1] (2) {};
      \node [circbutton, right of=2] (3) {PROGRAM};
      \node [circbutton, right of=3] (4) {PAGE 2};
      \node [selcircbutton, right of=4] (5) {SEQ.};
      \node [circbutton, right of=5] (6) {SETTINGS};
      \node [dummybutton, right of=6] (7) {};
      \node [dummybutton, right of=7] (8) {};

      \node [hcircbutton, below=0.1cm of 1] (9) {BACK};
      \node [hcircbutton, right of=9] (10) {SET};
      \node [dummybutton, right of=10] (11) {};
      \node [dummybutton, right of=11] (12) {};
      \node [dummybutton, right of=12] (13) {};
      \node [dummybutton, right of=13] (14) {};
      \node [hcircbutton, right of=14] (15) {UP};
      \node [hcircbutton, right of=15] (16) {DOWN};
    \end{tikzpicture}
  };
\end{tikzpicture}

\subsubsection{Entering or changing notes}
First enter the length you want for the note, using \btn{UP} and \btn{DOWN}. The length is given in a scale of 1/6 of the duration of one step, so if the length value is 3, that means that the note will last for half the duration of one step in the pattern. The length setting will be remembered the next time you enter a note again for the same channel.

When the length has been set, you can enter the note by either using MIDI, or by using the built-in keyboard, as shown in the figure above. As soon as a note has been entered, the sequencer will be return to the pattern editing mode.

When selecting notes on the built-in keyboard, you can change the current octave by holding \btn{OCTAVE} and using \btn{UP} and \btn{DOWN}.

\subsubsection{Clearing notes}
Press \btn{ERASE} if you want to clear the note. The sequencer will return to pattern editing mode, and the LED on the corresponding note will turn off to indicate that it is empty.

\subsubsection{Editing length only}
When the length has been set to the new value, press \btn{OK} to save the change without entering a new note.

\subsubsection{Aborting}
Press \btn{BACK} to go back to the pattern editing mode. Any changes you made to the note will be discarded.

\section{SETTINGS}

When in the settings mode, various aspects of how the \name operates can be changed. In this mode, the buttons have the following functions:

\begin{tikzpicture}[align=center, node distance=1.5cm]
  \node(scope1) {
    \begin{tikzpicture}
      \node [mhbutton] (1) {SQ1};
      \node [mhbutton, right of=1] (2) {SQ2};
      \node [mhbutton, right of=2] (3) {TRI};
      \node [mhbutton, right of=3] (4) {NOISE};
      \node [mhbutton, right of=4] (5) {DMC};
      \node [hbutton, right of=5] (6) {MIDI\\CHANNEL};
      \node [hbutton, right of=6] (7) {BATTERY\\VOLTAGE};
      \node [hbutton, right of=7] (8) {CHIP TYPE};

      \node [hbutton, below=1] (9) {PATCH\\RESET};
      \node [button, right of=9] (10) {};
      \node [button, right of=10] (11) {};
      \node [button, right of=11] (12) {};
      \node [button, right of=12] (13) {};
      \node [hbutton, right of=13] (14) {EXT.\\CLOCK};
      \node [hbutton, right of=14] (15) {SAMPLE\\RESET};
      \node [hbutton, right of=15] (16) {SAMPLE\\DELETE};
    \end{tikzpicture}
  };

  \node[at={($(scope1.south)$)},anchor=north] (scope2) {
    \begin{tikzpicture}
      \node [dummybutton] (1) {};
      \node [dummybutton, right of=1] (2) {};
      \node [circbutton, right of=2] (3) {PROGRAM};
      \node [circbutton, right of=3] (4) {PAGE 2};
      \node [circbutton, right of=4] (5) {SEQ.};
      \node [selcircbutton, right of=5] (6) {SETTINGS};
      \node [dummybutton, right of=6] (7) {};
      \node [dummybutton, right of=7] (8) {};

      \node [hcircbutton, below=0.1cm of 1] (9) {CLEAR};
      \node [hcircbutton, right of=9] (10) {SAVE};
      \node [dummybutton, right of=10] (11) {};
      \node [dummybutton, right of=11] (12) {};
      \node [dummybutton, right of=12] (13) {};
      \node [dummybutton, right of=13] (14) {};
      \node [hcircbutton, right of=14] (15) {UP};
      \node [hcircbutton, right of=15] (16) {DOWN};
    \end{tikzpicture}
  };
\end{tikzpicture}

\btn{UP} and \btn{DOWN} can be used to select a sample number. Sample numbers that are occupied are marked with a dot on the display. A selected sample can be deleted by pressing \btn{DELETE}.

\subsection{MIDI}

The \name can be controlled externally using MIDI.

\subsubsection{Assigning MIDI channels}

Each of the five sound channels can be assigned to any of the 16 MIDI channels, and will then only respond to incoming messages on the selected channel.

If you select the value 0, the sound channel will not listen to any MIDI channel.

To select a MIDI channel, enter the \lbl{SETTINGS} mode by pressing \btn{SETTINGS}. Hold down \btn{MIDI CHANNEL} and then press the desired channel's button. The LEDs will flash and you can select one of the 16 MIDI channels using the \btn{UP} and \btn{DOWN} buttons.

\subsubsection{External clock}
Use \btn{EXT. CLOCK} to use the incoming MIDI clock to set the tempo of the sequencer. When this is set to 1, the sequencer will not do anything unless an external MIDI device sends MIDI Clock messages. When the setting is 0, the internal tempo is used.

\subsection{Checking the battery voltage}
The NESIZER uses a battery for keeping the RAM storing the patches and samples alive when main power is disconnected. To check the battery's voltage, press and hold \btn{BATTERY VOLTAGE}. As long as the button is pressed, the battery's voltage will be shown in the display. If the battery voltage is below 2.6 V, it should be replaced. On startup, the NESIZER will give a warning if the battery is 2.5 V or less. The display will flash \verb+bL+ (Battery Low) for a short duration.

\subsection{Checking the 2A03 type}
To see which 2A03 chip is being used, press the LFO3 button. One of the following numbers will show up:
\begin{itemize}
\item 12: Genuine RP2A03
\item 15: 2A03 clone, e.g. 6527P
\item 16: 2A07 (PAL version of 2A03)
\end{itemize}

\subsection{Resetting patches}

Press \btn{PATCH RESET} to delete all patches. Every patch will be initialized to a basic patch with no channels enabled, square duty cycles set at 50\% and full envelope sustain levels with no attack, decay or release.

\subsection{Maintaining samples}

When in SETTINGS mode, the up and down buttons are used to select DMC samples. A dot appearing on the display indicates that the selected sample location is occupied. When an occupied sample is selected, it can be deleted by pressing \btn{SAMPLE DELETE}. All samples can be erased by pressing \btn{SAMPLE RESET}.

\subsection{Startup indicators}
If an error has occured, the display will flash some indicators to tell which errors were found during startup.
The errors that can occur are:
\begin{itemize}
\item \verb+bL+: Battery low (less than 2.5 V measured) - battery should be changed
\item \verb+cr+: Corrupt RAM - RAM contents will be re-initialized automatically
\end{itemize}


\section{MIDI}

\subsection{MIDI CC}
The Nesizer is capable of receiving MIDI Control Change (CC) messages. Each track can independently respond to CC messages, adjusting its parameters according to the specified ranges under PROGRAM and PAGE 2 modes. These ranges are automatically scaled to the full CC range of 0-127. Additionally, toggle-based ranges are mapped to 'OFF' for values below 64 and 'ON' for values 64 or higher.


\begin{tabular}{l | l | l | l | l | l }
  CC/(*) & SQ 1 & SQ 2 & Tri & Noise & DMC\\ \hline
  \textbf{1} & Duty & Duty & & & \\
  \textbf{5/(65)} & Glide & Glide & Glide & & \\
  \textbf{14} &  &  &  & Loop & Loop \\
  \textbf{30} & LFO 1 & LFO 1 & LFO 1 & LFO 1 & \\
  \textbf{31} & LFO 2 & LFO 2 & LFO 2 & LFO 2 & \\
  \textbf{32} & LFO 3 & LFO 3 & LFO 3 & LFO 3 & \\
  \textbf{72/(60)} & Release & Release &  & Release & \\
  \textbf{73/(61)} & Attack & Attack &  & Attack & \\
  \textbf{75/(62)} & Decay & Decay &  & Decay & \\
  \textbf{77} & Vol mod (LFO 3) & Vol mod (LFO 3) &  & Vol mod (LFO 3) & \\
  \textbf{79/(63)} & Sustain & Sustain &  & Sustain &\\
  \textbf{82} & Pitch Bend intensity & Pitch Bend intensity &  & Pitch Bend intensity & \\
  \textbf{85} & Detune & Detune & Detune &  &\\
  \textbf{95} & Octave Shift & Octave Shift & Octave Shift &  &\\
  \textbf{123} & Enable & Enable & Enable & Enable & Enable\\
\end{tabular}

* The number wrapped in '()' is the toggle message and will enable the parameter/disable the parameter.

\subsection{MIDI CC (Global)}

Global MIDI Control Change (CC) messages are received by all tracks and modify the global parameter value.

\begin{tabular}{l | l}
  CC & Global \\ \hline
  \textbf{50} & LFO 1 Period\\
    \textbf{51} & LFO 1 Waveform\\
    \textbf{52} & LFO 2 Period\\
    \textbf{53} & LFO 2 Waveform\\
    \textbf{54} & LFO 3 Period\\
    \textbf{55} & LFO 3 Waveform\\
\end{tabular}

\section{Using DMC samples}

\subsection{Sample upload workflow}

\tikzstyle{process} = [rectangle,centered,thick,draw=black]
\tikzstyle{arrow} = [thick,->,>=stealth]
\begin{tikzpicture}
  \begin{scope}
    \node (infile) {Input audio file};
    \node (convert) [process, below of=infile] {Convert sample to raw format};
    \node (sysex) [process, below of=convert] {Create SysEx wrapper};
    \node (sampleno) [right of=sysex, xshift=4cm] {Sample index number};
    \node (upload) [process, below of=sysex] {Send/play SysEx file on MIDI output device};

    \draw [arrow] (infile) -- (convert);
    \draw [arrow] (convert) -- (sysex);
    \draw [arrow] (sampleno) -- (sysex);
    \draw [arrow] (sysex) -- (upload);
  \end{scope}
\end{tikzpicture}

\subsection{Converting a sample file}
The first step of creating an audio sample suitable for DMC playback is to convert the audio file to \textbf{raw, 7-bits unsigned per sample, 16 kHz sample rate} format. Many tools will only allow conversion to an 8-bit per sample, however. It is therefore often necessary to convert to 8-bit first, and then truncate each sample to 7-bits when creating a System Exclusive file later.

A variety of audio software tools exist to perform this conversion. The following example uses the tool \emph{Sound eXchange (SoX)}, which is open source and available for Linux, Windows and macOS. Using SoX with an input audio file in \verb+.wav+ format, the conversion can be done as follows:

\verb+sox sample.wav -r 16000 -b 8 -e unsigned-integer sample.raw+

\subsection{Creating a SysEx file}
The converted sample needs to be wrapped with the necessary metadata and MIDI SysEx commands before it can be sent to the NESIZER via MIDI. The stream of bytes to send via MIDI must be in the following format:

\begin{verbatim}
  F0 (System Exclusive)
  7D (Sysex ID)
  NE (Device ID)
  00 (Sample load command)
  ii (Sample index)
  00 (Reserved)
  ss (Sample size 6..0)
  ss (Sample size 13..7)
  ss (Sample size 20..14)
  dd (Sample data 0)
  dd (Sample data 1)
  ..
  dd (Sample data N-1)
  F7 (End of Exclusive)
\end{verbatim}

\textbf{Note:} Any data byte contained within a MIDI System Exclusive message must never have the most significant bit set.

The sample index can be any number between 0 and 99. The index 0 corresponds to MIDI note 36, which is C3, commonly used for kick drum in the General MIDI drum map standard. This is the note that has to be played on the MIDI input or entered into a sequence pattern to trigger the sample on the DMC channel.

The sample size must be given in bytes, but split across three 7-bit words.

The sample data must be 7-bit raw sample values.

In the NESIZER source repository, there is a helpful tool \verb+gensysex.c+ which is a small C program to create this wrapper. This tool can be used as follows (assuming the tool is compiled into a binary called \verb+gensysex+):

\verb+gensysex sample <input format> <index> <raw filename> <sysex filename>+

The following example creates a SysEx file for the file \verb+sample.raw+ created in the above example. It places the sample at index 2, meaning that the note D3 must be played to trigger this sample:

\verb+gensysex sample 8bit 2 sample.raw sample.syx+

\subsection{Uploading samples}
The MIDI SysEx file containing the MIDI byte stream created in the previous step can now be played back on (or sent to) the MIDI adapter which the NESIZER is connected to. There are many different software options for doing this.

On a Linux system, the \verb+amidi+ ALSA utility tool is the easiest way to do this. To find connected MIDI devices, run \verb+amidi -l+ to list connected devices and make note of the Device identifier for the MIDI device you want to use. This should be of the form \verb+hw:x,x,x+, for example \verb+hw:2,0,0+.
To send the SysEx file out on the MIDI line from this MIDI device, run the following command:

\verb+amidi -p <MIDI device> -s <sysex file path>+

Continuing the above example, the \verb+sample.syx+ file can be sent to a NESIZER connected to MIDI device \verb+hw:2,0,0+ by running

\verb+amidi -p hw:2,0,0 -s sample.syx+

When this command is running, the connected NESIZER should be displaying the index number of the sample on the 7 segment display, and the 16 top LEDs should be lighting up a progress bar.

\end{document}
